%%%%%%%%%%%%%%%%%%%%%%%%%%%%%%%%%%%%%%%%%%%%%%%%%%%%%%%%%%%%%%%%%%%%%%%%%%%%%%%%%%%%%%%%%%%%%%%%%%%%%%%%%%%%%%%%%%%%%%%%%%%%%%%%%%%%%%%%%%%%
%%%%%%%%%%%%%%%%%%%%%%%%%%%%%%%%%%%%%%%%%%%%%%%%% Start Of the Designing of Passing Robot %%%%%%%%%%%%%%%%%%%%%%%%%%%%%%%%%%%%%%%%%%%%%%%%%%
%%%%%%%%%%%%%%%%%%%%%%%%%%%%%%%%%%%%%%%%%%%%%%%%%%%%%%%%%%%%%%%%%%%%%%%%%%%%%%%%%%%%%%%%%%%%%%%%%%%%%%%%%%%%%%%%%%%%%%%%%%%%%%%%%%%%%%%%%%%%
\section{Design of Passing Robot}
    \subsection{Overall Dimensions and Weight}
        \begin{table}[h]
            \caption{Dimensions and Weight of PR} \label{Dimensiona_PR} \small
            \begin{tabular}{|c|l|c|l|c|}
                \hline
                \textbf{Sr No}  & \textbf{Dimension} & \textbf{Value}  & \textbf{Dimension}      & \textbf{Value}   \\ \hline
                \textbf{1}      & Length             & \textbf{834} mm & Game length             & \textbf{1020} mm \\ \hline
                \textbf{2}      & Width              & \textbf{825} mm & Game Width              & \textbf{1034} mm \\ \hline
                \textbf{3}      & Height             & \textbf{843} mm & Game Height             & \textbf{843} mm  \\ \hline
                \textbf{4}      & \multicolumn{3}{c|}{Weight}                                    & \textbf{15} Kg      \\ \hline
            \end{tabular}
        \end{table}
        
    \subsection{Type Of Drive}
        The drive used for the locomotion of the robot is a \textbf{three-wheel holonomic drive} made using \textbf{Omni wheels}. Three-wheel Omni drive is 
        used because of its simplicity, agility and ease of control. Itis made using \textbf{Aluminium 6061 square tubes} because of its optimum 
        strength to weight ratio. For automating the three-wheel we have used an \textbf{Inertial Measurement Unit (IMU) MPU6050} used which allows 
        the robot to orient itself. For the distances feedback, we have used encoders with dummy wheels to \textbf{resolve locomotion along X and 
        Y-axis}.

    \subsection{Actuators and Sensors Integrated}
        \begin{itemize}
            \item Sensors
                \begin{table}[h]
                    \caption {Sensors Integrated on PR} \label{Sensors_PR}  \small
                    \begin{tabular}{|c|l|c|l|l|}
                        \hline  \hline
                        \textbf{Sr No}  & \textbf{Sensor}           & \textbf{Data Given}       & \textbf{Quantity} & \textbf{Information Deduced}                                          \\ \hline   \hline
                        \textbf{1}      & \textbf{Rotary Encoder}   & 600 Pulses Per Rotation   &        2          & The number of pulses obtained for calculation of                      \\
                                        &                           &                           &                   & corresponding PWM, this is done by using the map                      \\ 
                                        &                           &                           &                   & function.                                                             \\ \hline
                        \textbf{2}      & \textbf{MPU6050}          & Angular Velocity          &        1          & Angle of orientation of the robot                                     \\ \hline   \hline
                    \end{tabular}
                \end{table}
                
            \item Actuators
            \begin{table}[h]
                \caption {Actuators Used on PR} \label{Actuators_PR}  \small
                \begin{tabular}{|c|l|l|c|l|}
                    \hline  \hline
                    \textbf{Sr No}  & \textbf{Actuators}        & \textbf{Specifications}               & \textbf{Quantity}  & \textbf{Usage}                                               \\ \hline   \hline
                    \textbf{1}      & \textbf{Pneumatic Air}    & \textbf{12 mm} bore diameter,         & 1                  & For rotating the claw for a firm grip on the ball            \\
                                    & \textbf{Cylinder}         & \textbf{25 mm} stroke length.         &                    &                                                              \\ \hline
                    \textbf{2}      & \textbf{Pneumatic Air}    & \textbf{12 mm} bore diameter,         & 1                  & For linear displacement of the jig to slide the ball         \\
                                    & \textbf{Cylinder}         & \textbf{300 mm} stroke length.        &                    & in the rotors                                                \\ \hline                                  
                    \textbf{3}      & \textbf{PMDC Motor}       & \textbf{12 V DC},  \textbf{3000 RPM}, & 2                  & For rotating disc wheels                                     \\
                                    &                           & Torque: $\textbf{4 kg-cm}$            &                    &                                                              \\ \hline 
                    \textbf{4}      & \textbf{Planetary Geared} & \textbf{12 V DC},  \textbf{400 RPM},  & 2                  & For rotating gripper arms                                    \\
                                    & \textbf{PMDC Motor}       & Rated Torque: $\textbf{4 kg-cm}$      &                    &                                                              \\ \hline   \hline  
                \end{tabular}
            \end{table}
        \end{itemize}


    \subsection{Ball Picking and Passing Mechanism}
        \subsubsection{Ball Picking Mechainsm}
            The two \textbf{gripping arms made} from aluminium hollow square pipes are \textbf{coupled mechanically}. Aluminium material is used due to its 
            optimum strength to weight ratio. The part of the \textbf{concave curvature} is fixed at the end of one of the arms of gripper which holds 
            try ball firmly. To the same arm, a pneumatic piston is attached such that \textbf{extracting force} of it is used for gripping try ball. 
            Pneumatic air cylinder provides the appropriate amount of normal force such that the \textbf{frictional force} overcomes the \textbf{weight} of try 
            ball to grip it properly. The try ball gripped by the gripper is placed on the guideway with the help of two arms actuated using 
            \textbf{planetary geared PMDC motor} which aligns the ball with the help of \textbf{3D printed custom tool and rollers}. The guideway is set at an 
            angle of 45 0 to the ground. A pneumatic air cylinder of \textbf{300 mm} stroke length and \textbf{slider} are also mounted below the guideway. A slider 
            is connected to the piston rod. The guideway provides a \textbf{specific alignment} to the try ball such that the \textbf{major axis} of the ball is at 
            the desired angle to the rotating axis of the wheels which is required to pass the try ball to the TR.

        \subsubsection{Passing Mechanism}
            Two wheels are connected to the \textbf{12V DC motor} with the help of \textbf{female couplings}. Initially, 12V DC motors are rotated at optimal rpm, 
            then the pneumatic piston mounted on the structure of the guideway is actuated so that the try ball will be pushed in the space between 
            the two rotating wheels which will provide required \textbf{trajectory} to try ball for passing it to TR.\\
            • Torque Calculations required for motor selection:\\
            According to the dimensions of the parts involved in the gripping mechanism which are to be rotated by the DC geared motor and considering 
            their material and appropriate weight, it is calculated that their centre of mass occurs at a distance of \textbf{160 mm} from the centre of the shaft 
            of the motor. And the weight is \textbf{1007 grams}. With the safety factor of 1.5 and rounding up the values,\\
            $Torque = Weight \times Perpendicular distance = 1.505 \times 16 = \textbf{24.08 kg-cm}$\\
            So, considering this the motor used is of $\textbf{25 kg-cm}$ torque and $\textbf{400 rpm}$.

        \subsubsection{Working}

        \subsubsection{Control Stratergy}
            \begin{itemize}
                \item The encoders are coupled to the shaft of the motors. As soon as the motors are actuated, the encoders
                return the corresponding pulses to the microcontroller, which calculates the corresponding RPM of the
                motor.
                \item PID control strategy is implemented to maintain the desired RPM value.
                \item This process is repeated in a loop to get the stable system, and as soon as the required RPM for the
                motors is acquired, the piston is actuated to position the ball between the space, and a try is carried
                out.      
            \end{itemize}  

%%%%%%%%%%%%%%%%%%%%%%%%%%%%%%%%%%%%%%%%%%%%%%%%%%%%%%%%%%%%%%%%%%%%%%%%%%%%%%%%%%%%%%%%%%%%%%%%%%%%%%%%%%%%%%%%%%%%%%%%%%%%%%%%%%%%%%%%%%%%
%%%%%%%%%%%%%%%%%%%%%%%%%%%%%%%%%%%%%%%%%%%%%%%%%% End Of the Designing of Passing Robot %%%%%%%%%%%%%%%%%%%%%%%%%%%%%%%%%%%%%%%%%%%%%%%%%%%
%%%%%%%%%%%%%%%%%%%%%%%%%%%%%%%%%%%%%%%%%%%%%%%%%%%%%%%%%%%%%%%%%%%%%%%%%%%%%%%%%%%%%%%%%%%%%%%%%%%%%%%%%%%%%%%%%%%%%%%%%%%%%%%%%%%%%%%%%%%%