%%%%%%%%%%%%%%%%%%%%%%%%%%%%%%%%%%%%%%%%%%%%%%%%%%%%%%%%%%%%%%%%%%%%%%%%%%%%%%%%%%%%%%%%%%%%%%%%%%%%%%%%%%%%%%%%%%%%%%%%%%%%%%%%%%%%%%%%%%%%
%%%%%%%%%%%%%%%%%%%%%%%%%%%%%%%%%%%%%%%%%%%%%%%%% Start Of the Designing of Passing Robot %%%%%%%%%%%%%%%%%%%%%%%%%%%%%%%%%%%%%%%%%%%%%%%%%%
%%%%%%%%%%%%%%%%%%%%%%%%%%%%%%%%%%%%%%%%%%%%%%%%%%%%%%%%%%%%%%%%%%%%%%%%%%%%%%%%%%%%%%%%%%%%%%%%%%%%%%%%%%%%%%%%%%%%%%%%%%%%%%%%%%%%%%%%%%%%
\section{Design of Passing Robot}
    \subsection{Overall Dimensions and Weight}
        \begin{table}[h]
            \caption{Dimensions and Weight of PR} \label{Dimensions_PR} \small
            \begin{tabular}{|c|l|c|l|c|}
                \hline
                \textbf{Sr No}  & \textbf{Dimension} & \textbf{Value}  & \textbf{Dimension}      & \textbf{Value}   \\ \hline
                \textbf{1}      & Length             & \textbf{834} mm & Game length             & \textbf{1020} mm \\ \hline
                \textbf{2}      & Width              & \textbf{825} mm & Game Width              & \textbf{1034} mm \\ \hline
                \textbf{3}      & Height             & \textbf{843} mm & Game Height             & \textbf{843} mm  \\ \hline
                \textbf{4}      & \multicolumn{3}{c|}{Weight}                                    & \textbf{15} Kg      \\ \hline
            \end{tabular}
        \end{table}
        
    \subsection{Type Of Drive}
        The drive used for the locomotion of the robot is a \textbf{three-wheel holonomic drive} made using \textbf{Omni wheels}. Three-wheel Omni drive is 
        used because of its simplicity, agility and ease of control. It is made using \textbf{Aluminium 6061 square tubes} because of its optimum 
        strength to weight ratio. For automating the three-wheel we have used an \textbf{Inertial Measurement Unit (IMU) MPU6050} which allows 
        the robot to orient itself. For the distances feedback, we have used encoders with dummy wheels to \textbf{resolve locomotion along X and 
        Y-axis}.

    \subsection{Actuators and Sensors Integrated}
        \begin{itemize}
            \item Sensors
                \begin{table}[h]
                    \caption {Sensors Integrated on PR} \label{Sensors_PR}  \small
                    \begin{tabular}{|c|l|l|c|l|}
                        \hline  \hline
                        \textbf{Sr No}  & \textbf{Sensor}                   & \textbf{Specification}                & \textbf{Quantity} & \textbf{Usage}                                      \\ \hline   \hline
                        \textbf{1}      & \textbf{Incremental Optical}      & 600 Pulses Per Rotation               &        2          & To Control Position of the leg.                     \\
                                        & \textbf{Rotary Encoder}           &                                       &                   &                                                     \\ \hline         
                        \textbf{2}      & \textbf{Intertial Measurement}    & 3 - Axis Gyroscope                    &        1          & To Control Angular Position of  Robot.              \\ 
                                        & \textbf{Unit (IMU) MPU6050}       & 3 - Axis Accelerometer                &                   &                                                     \\ \hline                                     
                        \textbf{3}      & \textbf{Ultrasonic Sensor}        & \textbf{2 - 400cm} distance           &        1          & Distance calculation between ball rack and          \\ 
                                        & \textbf{(HC-SR04)}                & module, \textbf{0.3 cm} resolution    &                   & robot.                                              \\ \hline   \hline
                    \end{tabular}
                \end{table}
                
            \item Actuators
            \begin{table}[h]
                \caption {Actuators Used on PR} \label{Actuators_PR}  \small
                \begin{tabular}{|c|l|l|c|l|}
                    \hline  \hline
                    \textbf{Sr No}  & \textbf{Actuators}        & \textbf{Specifications}               & \textbf{Quantity}  & \textbf{Usage}                                               \\ \hline   \hline
                    \textbf{1}      & \textbf{Pneumatic Air}    & \textbf{8 mm} bore diameter,          & 1                  & For gripping the try ball.                                    \\
                                    & \textbf{Cylinder}         & \textbf{25 mm} stroke length.         &                    &                                                              \\ \hline
                    \textbf{2}      & \textbf{Pneumatic Air}    & \textbf{25 mm} bore diameter,         & 1                  & For linear displacement of the jig to slide the ball         \\
                                    & \textbf{Cylinder}         & \textbf{300 mm} stroke length.        &                    & in the rotors.                                                \\ \hline                                  
                    \textbf{3}      & \textbf{PMDC Motor}       & \textbf{12 V DC},  \textbf{3000 RPM}, & 2                  & For rotating disc wheels.                                     \\
                                    &                           & Torque: $\textbf{4 kg-cm}$            &                    &                                                              \\ \hline 
                    \textbf{4}      & \textbf{Planetary Geared} & \textbf{12 V DC},  \textbf{400 RPM},  & 2                  & For rotating gripper arms.                                    \\
                                    & \textbf{PMDC Motor}       & Rated Torque: $\textbf{4 kg-cm}$      &                    &                                                              \\ \hline   \hline   
                \end{tabular}
            \end{table}
        \end{itemize}


    \subsection{Ball Picking and Passing Mechanism}
        \subsubsection{Calculations}


        \subsubsection{Working}
            \textbf{A. Maintaining the desired distance from the ball rack with proper orientation:}                                   \\
            \textbf{MPU6050} is used for the orientation of pass Robot and for locomotion, Rotary encoder and MPU6050 feedback is used. \textbf{Ultrasonic sensor} 
            feedback is used for the detection of ball and distance calculation between ball rack and robot.                                                                                                                     \\
            \textbf{B. Gripping and Throwing a ball:}                                                                                  \\
            The gripping piston mechanism is used for gripping of the ball. The catapult piston mechanism is actuated for the
            throwing of the ball towards Try Robot. Range of Catapult mechanism is \textbf{7 m} to \textbf{7.5 m} height is \textbf{90 cm.}

%%%%%%%%%%%%%%%%%%%%%%%%%%%%%%%%%%%%%%%%%%%%%%%%%%%%%%%%%%%%%%%%%%%%%%%%%%%%%%%%%%%%%%%%%%%%%%%%%%%%%%%%%%%%%%%%%%%%%%%%%%%%%%%%%%%%%%%%%%%%
%%%%%%%%%%%%%%%%%%%%%%%%%%%%%%%%%%%%%%%%%%%%%%%%%% End Of the Designing of Passing Robot %%%%%%%%%%%%%%%%%%%%%%%%%%%%%%%%%%%%%%%%%%%%%%%%%%%
%%%%%%%%%%%%%%%%%%%%%%%%%%%%%%%%%%%%%%%%%%%%%%%%%%%%%%%%%%%%%%%%%%%%%%%%%%%%%%%%%%%%%%%%%%%%%%%%%%%%%%%%%%%%%%%%%%%%%%%%%%%%%%%%%%%%%%%%%%%%