%%%%%%%%%%%%%%%%%%%%%%%%%%%%%%%%%%%%%%%%%%%%%%%%%%%%%%%%%%%%%%%%%%%%%%%%%%%%%%%%%%%%%%%%%%%%%%%%%%%%%%%%%%%%%%%%%%%%%%%%%%%%%%%%%%%%%%%%%%%%
%%%%%%%%%%%%%%%%%%%%%%%%%%%%%%%%%%%%%%%%%%%%%%%%% Start Of the Designing of Trying Robot %%%%%%%%%%%%%%%%%%%%%%%%%%%%%%%%%%%%%%%%%%%%%%%%%%%
%%%%%%%%%%%%%%%%%%%%%%%%%%%%%%%%%%%%%%%%%%%%%%%%%%%%%%%%%%%%%%%%%%%%%%%%%%%%%%%%%%%%%%%%%%%%%%%%%%%%%%%%%%%%%%%%%%%%%%%%%%%%%%%%%%%%%%%%%%%%
\section{Design of Trying Robot}
    \subsection{Overall Dimensions and Weight}
        \begin{table}[h]
            \caption{Dimensions and Weight of TR} \label{Dimensions_TR} \small
            \begin{tabular}{|c|l|c|l|c|}
                \hline
                \textbf{Sr No}  & \textbf{Dimension} & \textbf{Value}  & \textbf{Dimension}      & \textbf{Value}   \\ \hline
                \textbf{1}      & Length             & \textbf{855} mm & Game length             & \textbf{1148} mm \\ \hline
                \textbf{2}      & Width              & \textbf{940} mm & Game Width              & \textbf{1045} mm \\ \hline
                \textbf{3}      & Height             & \textbf{848} mm & Game Height             & \textbf{848} mm  \\ \hline
                \textbf{4}      & \multicolumn{3}{c|}{Weight}                                    & \textbf{} Kg     \\ \hline
            \end{tabular}
        \end{table}

    \subsection{Type Of Drive}
        The base frame has been manufactured with \textbf{Aluminium 6061} hollow square pipes of \textbf{19 mm x 19 mm x 1.5 mm} cross-sectional thickness. 
        \textit{Aluminium ensures an efficient light-weight frame with uniform weight distribution}. Since Aluminium has low density ($2.7 g/cm^3$) compared 
        to other common metals available, it made the robot lightweight while ensuring swift translation and sturdiness of the base. For determination of orientaion 
        of robot, we have used \textbf{MPU6050 Inertial Measurement Unit} whereas \textbf{distance feedback taken using rotary encoder}. Being Holonomic 
        in nature it enables fast locomotion as well as quick response to the change in direction of travel. The three wheeled robot reduces the 
        overall weight of the robot and provides better load distribution over all the wheels.

    \subsection{Actuators and Sensors Integrated}
        \begin{itemize}
            \item Sensors
            \begin{table}[h]
                \caption {Sensors Integrated on TR} \label{Sensors_TR}  \small
                \begin{tabular}{|c|l|l|c|l|}
                    \hline  \hline
                    \textbf{Sr No}  & \textbf{Sensor}                   & \textbf{Specification}                & \textbf{Quantity} & \textbf{Usage}                                            \\ \hline   \hline                    
                    \textbf{1}      & \textbf{Incremental Optical}      & \textbf{600} Pulses Per Rotation      &        2          & To Control Position of the leg.                           \\
                                    & \textbf{Rotary Encoder}           &                                       &                   &                                                           \\ \hline         
                    \textbf{2}      & \textbf{Intertial Measurement}    & 3 - Axis Gyroscope                    &        3          & To Control Angular Position of  Robot and for distance.   \\ 
                                    & \textbf{Unit (IMU) MPU6050}       & 3 - Axis Accelerometer                &                   & distance .                                                \\ \hline 
                    \textbf{3}      & \textbf{Potentiometer}            & 10k linear taper 16 mm                &        2          & To control ball placing Flap Actuation.                   \\
                                    &                                   & rotary potentiometer                  &                   &                                                           \\ \hline
                    \textbf{4}      & \textbf{Digital Proximity}        & \textbf{0 - 80 cm} non contact type   &        1          & For detecting the kick ball.                              \\
                                    & \textbf{Sensor}                   & detection, \textbf{ 5V}               &                   &                                                           \\ \hline                                    
                    \textbf{5}      & \textbf{Ultrasonic Sensor}        & \textbf{2 - 400 cm} distance          &        1          & Distance calculation of ball rack base                    \\ 
                                    & \textbf{(HC-SR04)}                & module, \textbf{0.3 cm} resolution    &                   & frame.                                                    \\ \hline   \hline
                \end{tabular}
            \end{table}

            \item Actuators
            \begin{table}[h]
                \caption {Actuators Used on TR} \label{Actuators_TR}  \small
                \begin{tabular}{|c|l|l|c|l|}
                    \hline  \hline
                    \textbf{Sr No}  & \textbf{Actuators}        & \textbf{Specifications}               & \textbf{Quantity}  & \textbf{Usage}                                               \\ \hline   \hline
                    \textbf{1}      & \textbf{Johnson Geared}   & \textbf{12 V DC},  \textbf{60 RPM},   & 1                  & Used for actuating the placing flap.                         \\
                                    & \textbf{Motor}            & Rated Torque: $\textbf{20 kg-cm}$     &                    &                                                              \\ \hline 
                    \textbf{2}      & \textbf{Planetary Geared} & \textbf{12 V DC},  \textbf{850 RPM},  & 2                  & Required for actuating the kicking mechanism.                \\
                                    & \textbf{PMDC Motor}       & Stall Torque: $\textbf{45 kg-cm}$     &                    &                                                              \\ \hline   \hline  
                \end{tabular}
            \end{table}
        \end{itemize}


    \subsection{Ball Recieving Mechanism}
        The try ball receiving mechanism resembles a basket-like structure having trapezoid-shaped opening to receive the try ball. The receiving mechanism is covered with nylon fibre net to absorb the momentum 
        of the try ball. The structure is made using aluminium 6061 hollow square pipe since it provides optimum strength to the mechanism.
    
    \subsection{Try Mechainsm}
        The try mechanism is mounted just beneath the ball receiving mechanism. The try mechanism consists of a ramp, which is placed at an 
        inclination of \textbf{$32\degree$} to the ground and a flap which holds the ball inside of the try ball receiving mechanism. Once the try ball is received 
        in the mechanism with the help of the net, the ball will fall on the ramp and slide down through it till it is obstructed by the flap. 
        The rotation of the flap will stop at a point where the distance between the ramp and flap will be just enough to allow the ball to fall 
        in the try spot. At this point, there will be \textbf{contact between the flap and ball and the ball and ground as well}. While making a try, the 
        placing flap is actuated using Johnson geared motor. As the \textbf{ball must touch the robot and ground simultaneously}, the placing flap’s angular 
        movement must be controlled. A \textbf{10k Ohm potentiometer} coupled to the motor’s shaft is used as feedback to control the movement of placing 
        flap accurately.\\
        •  Alternate Gripping and Placing using Gripper:\\
            In case, \textbf{if the ball is not received in the basket} and falls in the receiving zone then the ball can be picked up using a pneumatic
            gripper. The gripper will grip the ball along the major axis and lift the ball using a \textbf{12V DC Johnson motor} with a \textbf{10k potentiometer}
            as \textbf{feedback}. Then the ball will be released by the gripper in the try spot.




%%%%%%%%%%%%%%%%%%%%%%%%%%%%%%%%%%%%%%%%%%%%%%%%%%%%%%%%%%%%%%%%%%%%%%%%%%%%%%%%%%%%%%%%%%%%%%%%%%%%%%%%%%%%%%%%%%%%%%%%%%%%%%%%%%%%%%%%%%%%
%%%%%%%%%%%%%%%%%%%%%%%%%%%%%%%%%%%%%%%%%%%%%%%%%% End Of the Designing of Trying Robot %%%%%%%%%%%%%%%%%%%%%%%%%%%%%%%%%%%%%%%%%%%%%%%%%%%%
%%%%%%%%%%%%%%%%%%%%%%%%%%%%%%%%%%%%%%%%%%%%%%%%%%%%%%%%%%%%%%%%%%%%%%%%%%%%%%%%%%%%%%%%%%%%%%%%%%%%%%%%%%%%%%%%%%%%%%%%%%%%%%%%%%%%%%%%%%%%